% !TEX encoding = UTF-8
\chapter{图片}
\echapter{pictures}


由于图片和表格处于浮动题环境内,初始看到请不要惊讶,后期完成后会自动调整,尽量不要用强制位置,容易造成页面的不协调,如果真的需要强制使用,可以使用captionof进行标号,但是非常不建议这样使用

\begin{figure}[!htbp]
	  \includegraphics[width=10.5cm]{canny.jpg}
	  \centering
	  \caption{Canny边缘提取算法创始人}
	  \label{canny}
\end{figure}



\begin{tcode}
	\begin{figure}[!htbp]
		\includegraphics[scale=0.5]{canny.jpg}
		\centering
		\caption{Canny边缘提取算法创始人}
		\label{canny}
	\end{figure}
\end{tcode}
如果想使用非浮动题环境或者软强制定位图片,在附加参数选项中加入[!htbp]或者使用captionof命令,例如:

{
	\centering
	\includegraphics[scale=0.5]{SDAU.jpg}
	\captionof{figure}{山东农业大学校徽}
	\label{xiaohui}
}

\begin{tcode}
	{
		\centering
		\includegraphics[scale=0.5]{SDAU.jpg}
		\captionof{figure}{山东农业大学校徽}
		\label{xiaohui}
	}
\end{tcode}
更多关于插图的技巧请自行阅读我在文件夹中留下的插图手册,或者自行观看工作室的直播。直播的链接为:
\href{https://www.bilibili.com/video/BV1nv41117q9}{https://www.bilibili.com/video/BV1nv41117q9}.
如果需要绘制矢量图,请自行阅读tikz帮助文档,或者在小屋内下载相关作者翻译的文档,或者使用ppt、visio等软件绘制


%tikz的工作示例
\begin{figure}
	\centering
	\begin{tikzpicture}
\node (ciji)  [] at (-0.5,0) {刺激};
\node (ganshouqi)[draw,minimum height = 0.8cm,thick] at (1.5,0)  {感受器};
\node (shenjing)    [draw,minimum height = 0.8cm,thick]  at (4,0) {神经网络};
\node (xiaoying)   [draw,minimum height = 0.8cm,thick] at (6.5,0) {效应器};
\node (xiangying) [minimum height = 0.8cm,thick] at (8.5,0) {响应};
		\draw [-stealth]  (ciji.east) -- (ganshouqi.west);
\draw [-stealth] ($(ganshouqi.east) + (0, 2mm)$) -- ($(shenjing.west) + (0,
2mm)$);
\draw [-stealth] ($(shenjing.west) + (0, -2mm)$) -- ($(ganshouqi.east) + (0,
-2mm)$);
\draw [-stealth] ($ (shenjing.east) + (0,2mm) $)  -- ($(xiaoying.west) + (0,2mm) $);
\draw [-stealth] ($ (xiaoying.west) + (0,-2mm) $)  -- ($(shenjing.east) + (0,-2mm) $);
\draw [-stealth] (xiaoying.east) -- (xiangying.west); 
	\end{tikzpicture}
   \caption{神经网络}
\end{figure}

%tikz的代码
\begin{tcode}
\begin{figure}
	\centering
	\begin{tikzpicture}
	\node (ciji)  [] at (-0.5,0) {刺激};
	\node (ganshouqi)[draw,minimum height = 0.8cm,thick] at (1.5,0)  {感受器};
	\node (shenjing)    [draw,minimum height = 0.8cm,thick]  at (4,0) {神经网络};
	\node (xiaoying)   [draw,minimum height = 0.8cm,thick] at (6.5,0) {效应器};
	\node (xiangying) [minimum height = 0.8cm,thick] at (8.5,0) {响应};
	\draw [-stealth]  (ciji.east) -- (ganshouqi.west);
	\draw [-stealth] ($(ganshouqi.east) + (0, 2mm)$) -- ($(shenjing.west) + (0,
	2mm)$);
	\draw [-stealth] ($(shenjing.west) + (0, -2mm)$) -- ($(ganshouqi.east) + (0,
	-2mm)$);
	\draw [-stealth] ($ (shenjing.east) + (0,2mm) $)  -- ($(xiaoying.west) + (0,2mm) $);
	\draw [-stealth] ($ (xiaoying.west) + (0,-2mm) $)  -- ($(shenjing.east) + (0,-2mm) $);
	\draw [-stealth] (xiaoying.east) -- (xiangying.west); 
	\end{tikzpicture}
   \caption{神经网络}
\end{figure}

\end{tcode}
