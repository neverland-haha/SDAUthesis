% !TEX encoding = UTF-8
%请勿忘记添加echapter
\chapter{注意事项}
\echapter{notice}

\begin{enumerate}
	\item 本模板已经经过刘梦良老师和时彬彬老师的检查,基本可以实现所需要的效果。
	\item \LaTeX 可以解决跨系统时word格式会改变的问题,同时也能在最大程度上减少老师修改格式的压力。
	\item 由于word和\LaTeX 固有的差别,不可能$100$\% 复现word效果。
	\item 创建了一个山东农业大学\LaTeX 交流群,群号为835684647,欢迎感兴趣的老师同学加入。
	\item 如果有什么多余需要和错误改正,请加入交流群联系作者。
	\item  请自行参阅压缩包内的相关文档,文档所提供的帮助比百度会大的多。请耐心阅读英文文档。
	\item 作者的CSDN: \href{https://blog.csdn.net/weixin_43342986}{https://blog.csdn.net/weixin\_43342986}内有一些\LaTeX 的内容。
	\item 作者的github: \href{https://github.com/neverland-haha?tab=repositories}{https://github.com/neverland-haha?tab=repositories}里面也可以下载到相关内容,也可以留下issues。
	\item 请特别注意首页题目过长时的格式修改问题,如果不会修改可以咨询作者,同时也请注意最后致谢时的行距问题。
	\item 非常欢迎更加厉害的道友能够精简或者修改我的代码,为学校的\LaTeX 模板出一份力,为学术与科研助一份力。
	\item 祝大家能够快乐的使用\LaTeX , Happy \LaTeX ing!
\end{enumerate}

\textcolor{red}{version3.0相比version2.0以及小屋内的代码的改进:}
\begin{enumerate}
	\item 添加更为详细的工作示例代码。
	\item 修改了行距等问题
	\item 修改了存在的一些字体大小的问题。
	\item 在文件夹中加入了一些基础类教程的文档,更有利于人们学习相关内容。
\end{enumerate}