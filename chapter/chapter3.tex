% !TEX encoding = UTF-8
\chapter{表格}
\echapter{chart}

由于格式要求表格字体内为5号字,所以需要自己在table和tabular环境前加上 \verb|\zihao{5}| \\的代码,可以参照工作示例。
表格可以使用浮动体或者非浮动体的方式,建议没有强迫症的同学还是使用浮动体的环境,如果有强迫症的同学也还是稍微改一改,如果非要选择非浮动题环境,那么参考下面的工作示例。

\begin{tcode}
%浮动题环境示例
\begin{table}
	\zihao{5}
	\centering
	\caption{已经定义了的定理}
	\begin{tabular}{|c|c|}
		\hline               
		定义      &   definition  \\   \hline 
		定理      &   theorem    \\   \hline
		公理      &   axiom		\\   \hline
		引理		&	lemma		\\   \hline
		命题	   &	proposition \\  \hline
		注		  &	  remark		\\  \hline
		解		  &	  solution   \\ \hline
		证明		&	proofname  \\ \hline
		
	\end{tabular}
\end{table}
\end{tcode}

%浮动题环境示例
\begin{table}
	\zihao{5}
	\centering
	\caption{已经定义了的定理}
	\begin{tabular}{|c|c|}
		\hline               
		定义      &   definition  \\   \hline 
		定理      &   theorem    \\   \hline
		公理      &   axiom		\\   \hline
		引理		&	lemma		\\   \hline
		命题	   &	proposition \\  \hline
		注		  &	  remark		\\  \hline
		解		  &	  solution   \\ \hline
		证明		&	proofname  \\ \hline
	\end{tabular}
\end{table}


%非浮动题工作代码示例。
%由于captionof针对非浮动体,所以它距离所需要的表或者线的垂直距离变成了0,由于book、article等类的间距为10pt所以需要加入vspace=10pt
\begin{tcode}
	{%非浮动题工作代码示例。
		\zihao{5}	
		\centering
		\captionof{table}{已经定义了的定理}
		\vspace{10pt}				
		\begin{tabular}{|c|c|}
			\hline               
			定义      &   definition  \\   \hline 
			定理      &   theorem    \\   \hline
			公理      &   axiom		\\   \hline
			引理		&	lemma		\\   \hline
			命题	   &	proposition \\  \hline
			注		  &	  remark		\\  \hline
			解		  &	  solution   \\ \hline
			证明		&	proofname  \\ \hline
				\end{tabular}
	}
\end{tcode}

%非浮动题环境
	{
	\zihao{5}
	\centering
	\captionof{table}{已经定义了的定理}
	\vspace{10pt}
	\begin{tabular}{|c|c|}
		\hline               
		定义      &   definition  \\   \hline 
		定理      &   theorem    \\   \hline
		公理      &   axiom		\\   \hline
		引理		&	lemma		\\   \hline
		命题	   &	proposition \\  \hline
		注		  &	  remark		\\  \hline
		解		  &	  solution   \\ \hline
		证明		&	proofname  \\ \hline
	\end{tabular}

}


